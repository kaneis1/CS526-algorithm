\documentclass{article}
\usepackage{amsmath}
\usepackage{amsfonts}
\usepackage{algorithm}
\usepackage{algpseudocode}

\title{CS526-HW3}
\author{Zongqi Cui}
\date{October 2024}

\begin{document}

\maketitle

\section*{Problem 1}

According to the problem, we can build the payoff matrix as follows:
\[
\begin{pmatrix}
3 & 1 & 1 & 1 \\
\frac{3}{2}  & 2 & 2 & 1 \\
1 & \frac{4}{3} & \frac{5}{3} & 1 \\
1 & \frac{4}{3} & \frac{5}{3} & 2 \\
\end{pmatrix}
\]

(a) If we are the first to announce the strategy, we must have \( z = \max_d \mathbb{E}_t[r(t, d)] \). Therefore, we have the following system of inequalities:

\[
\begin{aligned}
&\max z \\
& z \geq 3q_1 + q_2 + q_3 + q_4 \\
& z \geq \frac{3}{2}q_1 + 2q_2 + q_3 + q_4 \\
& z \geq q_1 + \frac{4}{3}q_2 + \frac{5}{3}q_3 + q_4 \\
& z \geq q_1 + \frac{4}{3}q_2 + \frac{5}{3}q_3 + 2q_4 \\
& q_1 + q_2 + q_3 + q_4 = 1
\end{aligned}
\]

We divide \( z \) for all the equations and substitute \( q_i \)'s with \( x_i \)'s, resulting in the system:
\[
\begin{aligned}
1 &\geq 3x_1 + x_2 + x_3 + x_4 \\
1 &\geq \frac{3}{2}x_1 + 2x_2 + x_3 + x_4 \\
1 &\geq x_1 + \frac{4}{3}x_2 + \frac{5}{3}x_3 + x_4 \\
1 &\geq x_1 + \frac{4}{3}x_2 + \frac{5}{3}x_3 + 2x_4
\end{aligned}
\]

Using the \texttt{optimize} function from the \texttt{scipy} package to solve this LP problem:

\begin{verbatim}
from scipy import optimize as opt
import numpy as np

c = np.array([1, 1, 1, 1])
a = np.array([[3, 1, 1, 1], [1.5, 2, 1, 1], [1, 4/3, 5/3, 1], [1, 4/3, 5/3, 2]])
b = np.array([1, 1, 1, 1])

ans = opt.linprog(-c, a, b)
p = 1 / sum(ans['x'])
ans_list = [i * p for i in ans['x']]
print(ans_list)
print(sum(ans_list))
\end{verbatim}

The probabilities for choosing each number are:
\[
p_1 \approx 21.05\%, \quad p_2 \approx 31.58\%, \quad p_3 \approx 47.37\%, \quad p_4 \approx 0\%
\]

we can find that $min_P = 1.42$.

(b) For \( \min_d \mathbb{E}_d[r(t, d)] \), after transformation, we get:
\[
\begin{aligned}
1 &\leq 3x_1 + \frac{3}{2}x_2 + x_3 + x_4 \\
1 &\leq x_1 + 2x_2 + \frac{4}{3}x_3 + \frac{4}{3}x_4 \\
1 &\leq x_1 + x_2 + \frac{5}{3}x_3 + \frac{5}{3}x_4 \\
1 &\leq x_1 + x_2 + x_3 + 2x_4
\end{aligned}
\]

Using similar code as above, the probabilities are:
\[
q_1 \approx 0\%, \quad q_2 \approx 0\%, \quad q_3 \approx 1\%, \quad q_4 \approx 0\%
\]
\[P=\frac{5}{3}\]
\section*{Problem 2}

We aim to avoid two cases:
\begin{itemize}
    \item \textbf{Bad1}: Some points are not covered. \( \Pr[\text{Bad1}] \geq n \cdot \Pr[\text{a is not covered}] = \left(\frac{1}{e}\right)^c \)
    \item \textbf{Bad2}: The cost \( C \) is too large.
    \[
    \Pr[\text{Bad2}] = \Pr[\text{cost}(C) > (1 + \epsilon)(\ln n)\cdot \text{OPT}_f] \leq \frac{\ln n + c}{(1 + \epsilon) \cdot \ln n}
    \]
\end{itemize}

We want \( \Pr[\text{Bad1}] \) to be small and \( \Pr[\text{Bad2}] \) close to 1. Set:
\[
\Pr[\text{Bad1}] < \frac{\epsilon}{8}, \quad \Pr[\text{Bad2}] < 1 - \frac{\epsilon}{4}
\]
Thus, \( \Pr[\text{Success}] > \frac{\epsilon}{8} \). After repeating the process \( \frac{8}{\epsilon} \) times:
\[
\Pr[\text{no success}] < \left(1 - \frac{\epsilon}{8}\right)^{\frac{8}{\epsilon}} \leq \frac{1}{e} < \frac{1}{2}
\]
Therefore, there is at least a 50\% chance of finding a successful \( C \).

\section*{Problem 3}
(a) Let $A \subseteq B \subseteq \{1, 2, \dots, m\}$ be two sets of indices and let $x \notin B$. We need to show that:
\[
f(A \cup \{x\}) - f(A) \geq f(B \cup \{x\}) - f(B)
\]
The function $f(I) = \left| \bigcup_{i \in I} S_i \right|$ gives the size of the union of sets $S_i$ indexed by $I$. This implies that $f$ measures the number of distinct elements in the union of the sets.

Now, consider the following:
- $f(A \cup \{x\}) - f(A)$ represents the number of new elements introduced by adding the set $S_x$ to the union of sets indexed by $A$.
- Similarly, $f(B \cup \{x\}) - f(B)$ represents the number of new elements introduced by adding $S_x$ to the union of sets indexed by $B$.

Since $A \subseteq B$, the union $\bigcup_{i \in B} S_i$ already contains at least as many elements as $\bigcup_{i \in A} S_i$. Therefore, the contribution of $S_x$ (i.e., the number of new elements introduced by adding $S_x$) will be greater when added to $A$ than when added to $B$. In other words:
\[
f(A \cup \{x\}) - f(A) \geq f(B \cup \{x\}) - f(B)
\]
Thus, the function $f$ satisfies the diminishing returns property, and we conclude that $f$ is submodular.


(b) The idea behind the code is simple. For each iteration, union each item in set \( T \) with the previous result and choose the maximum combination. Repeat \( K \) times to ensure there are \( K \) items in the final result.

(c) In the worst case, function \( f \) needs to be called \( O(m \cdot k) \) times.

\begin{algorithm}
\caption{Greedy Algorithm for Problem 3.b}
\begin{algorithmic}
\State $T = S_1 + S_2 + ... + S_i$
\State $solu \gets 0$
\State $solu\_set \gets \emptyset$
\State $curr\_max \gets \emptyset$
\For{$i \gets 1, k$}
    \For{$t \in T_1, T_{\text{last}}$}
        \If{$solu \leq f(solu\_set \cup t)$}
            \State $curr\_max \gets t$
        \EndIf
    \EndFor
    \State $solu \gets f(solu\_set \cup curr\_max)$
    \State $solu\_set \gets solu\_set \cup curr\_max$
    \State $curr\_max \gets \emptyset$
    \State $T \gets T \setminus curr\_max$
\EndFor
\end{algorithmic}
\end{algorithm}

(d) Set \( n_t = \text{OPT}_k - f(I) \) after \( t \) iterations. Since the covered numbers are in \( \text{OPT}_k \), there must be \( n_t/k \) sets. Hence:
\[
n_{t+1} \leq n_t - \frac{n_t}{k}
\]
Thus:
\[
n_t \leq \text{OPT}_k \cdot \left(1 - \frac{1}{k}\right)^t
\]
Transforming, we get:
\[
f(I) \geq \left[1 - \left(1 - \frac{1}{k}\right)^k\right] \cdot \text{OPT}_k
\]

\section*{Problem 4}

(a) Let \( x_i \) represent choosing an actor and \( y_j \) represent choosing an investor. The profit for each investor is represented as \( y_j p_j - x_i s_i \). The ILP is formulated as:
\[
\max \left( \sum y_j p_j - \sum x_i s_i \right)
\]
subject to:
\[
y_j \leq x_i \quad \forall j \in [1, n], i \in L_j, \quad x_i, y_j \in \{0, 1\}
\]

(b) If there is a non-integer solution, a better integer solution exists. If all \( x_i < 1 \) and the total profit is negative, set \( x_i = 0 \). If \( x_i \) is between 0 and 1, let \( r = \max x_i \) and multiply \( x_i \) by \( 1/r \) until all \( x_i = 1 \).

\section*{Problem 5}

(a) Start with all vertices having a “fattest path capacity” of 0, except for the source s, which starts with infinity (to represent no restriction at the beginning).\\
For each vertex v, try to relax the fattest path by updating the capacity of its neighbors along the edges.\\
Use a max-priority queue (or a binary heap) to always expand the vertex with the largest current path capacity.\\
The algorithm terminates when the sink t is reached.
In Dijkstra’s algorithm, we minimize the distance (or cost) to each vertex, while in this algorithm, we maximize the capacity of the smallest edge along a path. The priority is based on capacity, not cost.\\

(b) To find the maximum flow of a graph \( G \), this can be transformed into finding the minimum cut. The number of edges in the minimum cut is at most \( |E| \), and the maximum flow is the sum of edges in the minimum cut. So we can conclude that the maximum flow for $G$ will have at most \( |E| \) edges.

(c) The number of edges in the maximum flow cannot exceed \( |E| \). After increasing flow \( F \), we could get a equation like this 
    \[F_{t+1}\leq F_t-\frac{F_t}{|E|}\]
, and the run time is \( O(|E| \cdot \log F) \).

\end{document}
